\documentclass[a4paper]{article}
\usepackage[hidelinks]{hyperref}
\usepackage{fontspec,xltxtra,xunicode}
\usepackage[a4paper,top=0.9in,bottom=0.7in,left=1.3in,right=1.3in,includehead,includefoot]{geometry}

\setmainfont{Ubuntu}

\usepackage{chngcntr}
\counterwithout{subsection}{section}

\renewcommand*{\thesection}{}

\let\emph\relax
\DeclareTextFontCommand{\emph}{\bfseries}

\begin{document}
\title{By Laws of Couchers, Inc.}
\date{April 24, 2024}
\maketitle

\section*{ARTICLE I. NAME OF ORGANIZATION}

The name of the corporation is Couchers, Inc.

\section*{ARTICLE II. CORPORATE PURPOSE}


\subsection*{Section 1. Nonprofit Purpose}

This corporation is organized exclusively for charitable and cultural purposes, including, for such purposes, the making of distributions to organizations that qualify as exempt organizations under section 501(c)(3) of the Internal Revenue Code, or the corresponding section of any future federal tax code.

\subsection*{Section 2. Specific Purpose}

Couchers, Inc. is a non-profit organization dedicated to promoting and funding projects that advance cultural exchange and understanding.

\section*{ARTICLE III. MEMBERSHIP}

The membership of the corporation shall consist of the members of the Board of Directors.

\subsection*{Section 1. Eligibility for Membership}

Application for voting membership shall be open to natural persons that support the purpose statement in Article II, Section 2. All memberships shall be granted upon a majority vote of the board. Membership is granted after completion and receipt of a membership application, annual dues, and approval by the board.

\subsection*{Section 2. Annual Dues}

The amount required for annual dues shall be specified by a meeting of the full membership. Continued membership is contingent upon being up-to-date on membership dues.

\subsection*{Section 3. Rights of Members}

Each member shall be eligible to appoint one voting representative to cast the member's vote in association elections.

\subsection*{Section 4. Resignation and Termination}

Any member may resign by filing a written resignation with the secretary. Resignation shall not relieve a member of unpaid dues, or other charges previously accrued. A member can have their membership terminated by a majority vote of the membership.

\subsection*{Section 5. Non-voting Membership}

The board shall have the authority to establish and define non-voting categories of membership.

\subsection*{Section 6. Ceasing of Membership}

A person immediately stops being a member if they die, resign by writing to the secretary, are removed by a resolution of the directors, or have not responded within three months to a written request from the secretary that they confirm in writing that they want to remain a member.

\section*{ARTICLE IV. MEETINGS OF MEMBERS }


\subsection*{Section 1. Special Meetings}

Special meetings may be called by the chair, the Executive Committee, or a simple majority of the board of directors. A petition signed by one third of voting members may also call a special meeting.

\subsection*{Section 2. Quorum}

A quorum for a meeting of the members shall consist of at least two thirds of the active membership. No business may be conducted if a quorum is not present. If there is no quorum present within 30 minutes after the time stated in the notice of the meeting, the meeting is adjourned to the place, date and time that the chair specifies, else exactly one week later in the same place and time if the chair does not specify.

\subsection*{Section 3.  Voting}

All issues to be voted on shall be decided by a simple majority of those present at the meeting in which the vote takes place.

\subsection*{Section 4. Notice}

Notice of a meeting of members shall be provided in writing to each director and each member entitled to vote at that meeting at least 14 days before the meeting, or less than 14 days if agreed to by members with 95\% of votes. The notice shall include the place, date and time for the meeting, and if the meeting is to be held in two or more places, the technology that will be used to facilitate this.

\section*{ARTICLE V. BOARD OF DIRECTORS}


\subsection*{Section 1. General Powers}

The affairs of the Corporation shall be managed by its Board of Directors.  The Board of Directors shall have control of and be responsible for the management of the affairs and property of the Corporation.

\subsection*{Section 2. Number, Tenure, Requirements, and Qualifications}

The number of Directors shall be fixed from time-to-time by the Directors but shall be no less than three (3), and nor more than fifteen (15). The members of the Board of Directors shall, upon election, immediately enter upon the performance of their duties and shall continue in office until their successors shall be duly elected and qualified.  All members of the Board of Directors and Advisory Council must be approved by a majority vote of the members present and voting.  No vote on new members of the Board of Directors, or Advisory Council, shall be held unless a quorum of the Board of Directors is present as provided in Section 6 of this Article.

No two members of the Board of Directors related by blood or marriage/domestic partnership within the second degree of consanguinity or affinity may serve on the Board of Directors at the same time, unless a resolution is passed by the Board to make an exception.

Each member of the Board of Directors shall be a member of the Corporation whose membership dues are paid in full and shall hold office for up to a three-year term as submitted by the nominations committee.

\subsection*{Section 3. Regular and Annual Meetings}

An annual meeting of the Board of Directors shall be held each calendar year. The time, date and location shall be resolved by the Executive Committee of the Board of Directors.  Notice of these meetings shall be sent to all members of the Board of Directors no less than ten (10) days, prior to the meeting date.

\subsection*{Section 4. Special Meetings}

Special meetings of the Board of Directors may be called by or at the request of the President or any two members of the Board of Directors.  The person or persons authorized to call special meetings of the Board of Directors may fix any location, as the place for holding any special meeting of the Board called by them.

\subsection*{Section 5. Notice}

Notice of any special meeting of the Board of Directors shall be given at least two (2) days in advance of the meeting by telephone, facsimile or electronic methods or by written notice.  Any Director may waive notice of any meeting.  The attendance of a Director at any meeting shall constitute a waiver of notice of such meeting, except where a Director attends a meeting for the express purpose of objecting to the transaction of any business because the meeting is not lawfully called or convened.  Neither the business to be transacted at, nor the purpose of, any regular meeting of the Board of Directors need be specified in the notice or waiver of notice of such meeting, unless specifically required by law or by these by-laws.
Written notice in all cases in these bylaws includes email correspondence.

\subsection*{Section 6. Quorum}

The presence, in person, of a majority of current members of the Board of Directors shall be necessary at any meeting to constitute a quorum to transact business, but a lesser number shall have power to adjourn to a specified later date without notice.  The act of a majority of the members of the Board of Directors present at a meeting at which a quorum is present shall be the act of the Board of Directors, unless the act of a greater number is required by law or by these by-laws.
Meetings may occur in physical presence, by teleconference or similar technology.

\subsection*{Section 7. Forfeiture}

Any member of the Board of Directors who fails to fulfill any of their requirements as set forth in Section 2 of this Article by September 1st shall automatically forfeit their seat on the Board.  The Secretary shall notify the Director in writing that their seat has been declared vacant, and the Board of Directors may forthwith immediately proceed to fill the vacancy.  Members of the Board of Directors who are removed for failure to meet any or all of the requirements of Section 2 of this Article are not entitled to vote at the annual meeting and are not entitled to the procedure outlined in Section 14 of this Article in these by-laws.

\subsection*{Section 8. Vacancies}

Whenever any vacancy occurs in the Board of Directors it shall be filled without undue delay by a majority vote of the remaining members of the Board of Directors at a regular meeting.  Vacancies may be created and filled according to specific methods approved by the Board of Directors.

\subsection*{Section 9. Compensation}

Members of the Board of Directors shall not receive any compensation for their services as Directors.

\subsection*{Section 10. Informal Action by Directors}

Any action required by law to be taken at a meeting of the Directors, or any action which may be taken at a meeting of Directors, may be taken without a meeting if a consent in writing, setting forth the action so taken, shall be signed by two-thirds (2/3) of all of the Directors following notice of the intended action to all members of the Board of Directors.

\subsection*{Section 11. Confidentiality}

Directors shall not discuss or disclose information about the Corporation or its activities to any person or entity unless such information is already a matter of public knowledge, such person or entity has a need to know, or the disclosure of such information is in furtherance of the Corporations' purposes, or can reasonably be expected to benefit the Corporation.  Directors shall use discretion and good business judgment in discussing the affairs of the Corporation with third parties.  Without limiting the foregoing, Directors may discuss upcoming fundraisers and the purposes and functions of the Corporation, including but not limited to accounts on deposit in financial institutions.

Each Director shall execute a confidentiality agreement consistent herewith upon being voted onto and accepting appointment to the Board of Directors.

\subsection*{Section 12. Advisory Council}

An Advisory Council may be created whose members shall be elected by the members of the Board of Directors annually but who shall have no duties, voting privileges, nor obligations for attendance at regular meetings of the Board.  Advisory Council members may attend said meetings at the invitation of a member of the Board of Directors.  Members of the Advisory Council shall possess the desire to serve the community and support the work of the Corporation by providing expertise and professional knowledge.  Members of the Advisory Council shall comply with the confidentiality policy set forth herein and shall sign a confidentiality agreement consistent therewith upon being voted onto and accepting appointment to the Advisory Council.

\subsection*{Section 13. Parliamentary Procedure}

Any question concerning parliamentary procedure at meetings shall be determined by the President by reference to Robert's Rules of Order.

\subsection*{Section 14. Removal}

Any member of the Board of Directors or members of the Advisory Council may be removed with or without cause, at any time, by vote of three-quarters (3/4) of the members of the Board of Directors if in their judgment the best interest of the Corporation would be served thereby.  Each member of the Board of Directors must receive written notice of the proposed removal at least ten (10) days in advance of the proposed action.  An officer who has been removed as a member of the Board of Directors shall automatically be removed from office.

Members of the Board of Directors who are removed for failure to meet the minimum requirements in Section 2 of this Article in these by-laws automatically forfeit their positions on the Board pursuant to Section 7 of this Article, and are not entitled to the removal procedure outlined in Section 14 of this Article.

\subsection*{Section 15. Founding Directors}

The corporation acknowledges the special role of the two founders of the corporation, Aapeli Vuorinen and Itsi Weinstock, collectively the Founding Directors. To ensure the long-term stability and direction of the project, the following special rules apply to the Founding Directors

\begin{enumerate}
\item The term of a Founding Director is not limited by Section 2 of this Article, and a Founding Director can not be removed according to the procedures in Section 14 of this Article. A Founding Director may remain a Director on the Board of Directors as long as they see fit, only ceasing to be a Director upon their resignation.
\item In the event that the Board has five (5) or fewer Directors, or a meeting occurs with five (5) or fewer members present, the following provisions shall apply: for all votes, decisions, and calculations (including those for quorum, majority, and any other decision where each Director--present or not--is counted); Itsi Weinstock's voting rights or count shall be assigned to, exercised by, and counted for Aapeli Vuorinen.
\end{enumerate}

\section*{ARTICLE VI. OFFICERS}

The Board of Directors may appoint and remove Officers.

\section*{ARTICLE VII. COMMITTEES}


\subsection*{Section 1. Committee Formation}

The board may create committees as needed, such as fundraising, public relations, data collection, etc. The board chair appoints all committee chairs.

\subsection*{Section 2. Executive Committee}

The four officers serve as the members of the Executive Committee. Except for the power to amend the Articles of Incorporation and Bylaws, the Executive Committee shall have all the powers and authority of the board of directors in the intervals between meetings of the board of directors, and is subject to the direction and control of the full board.

\subsection*{Section 3. Finance Committee}

The treasurer is the chair of the Finance Committee, which includes three other board members. The Finance Committee is responsible for developing and reviewing fiscal procedures, fundraising plans, and the annual budget with staff and other board members. The board must approve the budget and all expenditures must be within budget. Any major change in the budget must be approved by the board or the Executive Committee. The fiscal year shall be the calendar year. Annual reports are required to be submitted to the board showing income, expenditures, and pending income. The financial records of the organization are public information and shall be made available to the membership, board members, and the public.

\section*{ARTICLE VIII. CORPORATE STAFF}


\subsection*{Section 1: Executive Director}

The Board of Directors may hire an Executive Director who shall serve at the will of the Board.  The Executive Director shall have immediate and overall supervision of the operations of the Corporation, and shall direct the day-to-day business of the Corporation, maintain the properties of the Corporation, hire, discharge, and determine the salaries and other compensation of all staff members under the Executive Director's supervision, and perform such additional duties as may be directed by the Executive Committee or the Board of Directors.  No officer, Executive Committee member or member of the Board of Directors may individually instruct the Executive Director or any other employee.  The Executive Director shall make such reports at the Board and Executive Committee meetings as shall be required by the President or the Board.  The Executive Director shall be an ad-hoc member of all committees.
The Executive Director may not be related by blood or marriage/domestic partnership within the second degree of consanguinity or affinity to any member of the Board of Directors or Advisory Council.  The Executive Director may be hired at any meeting of the Board of Directors by a majority vote and shall serve until removed by the Board of Directors upon an affirmative vote of three-quarters (3/4) of the members present at any meeting of the Board Directors.  Such removal may be with or without cause.  Nothing herein shall confer any compensation or other rights on any Executive Director, who shall remain an employee terminable at will, as provided in this Section.

\section*{ARTICLE IX. CONFLICT OF INTEREST AND COMPENSATION}


\subsection*{Section 1: Purpose}

The purpose of the conflict of interest policy is to protect this tax-exempt organization's (Organization) interest when it is contemplating entering into a transaction or arrangement that might benefit the private interest of an officer or director of the Organization or might result in a possible excess benefit transaction. This policy is intended to supplement but not replace any applicable state and federal laws governing conflict of interest applicable to nonprofit and charitable organizations.

\subsection*{Section 2: Definitions}

\begin{enumerate}
\item \emph{Interested Person}: Any director, principal officer, or member of a committee with governing board delegated powers, who has a direct or indirect financial interest, as defined below, is an interested person.
\item \emph{Financial Interest}: A person has a financial interest if the person has, directly or indirectly, through business, investment, or family:
\begin{enumerate}
\item An ownership or investment interest in any entity with which the Organization has a transaction or arrangement,
\item A compensation arrangement with the Organization or with any entity or individual with which the Organization has a transaction or arrangement, or
\item A potential ownership or investment interest in, or compensation arrangement with, any entity or individual with which the Organization is negotiating a transaction or arrangement.
\end{enumerate}
\end{enumerate}


Compensation includes direct and indirect remuneration as well as gifts or favors that are not insubstantial.

A financial interest is not necessarily a conflict of interest. Under Article III, Section 2, a person who has a financial interest may have a conflict of interest only if the appropriate governing board or committee decides that a conflict of interest exists.

\subsection*{Section 3. Procedures}

\begin{enumerate}
\item \emph{Duty to Disclose}. In connection with any actual or possible conflict of interest, an interested person must disclose the existence of the financial interest and be given the opportunity to disclose all material facts to the directors and members of committees with governing board delegated powers considering the proposed transaction or arrangement.
\item \emph{Determining Whether a Conflict of Interest Exists}. After disclosure of the financial interest and all material facts, and after any discussion with the interested person, they shall leave the governing board or committee meeting while the determination of a conflict of interest is discussed and voted upon. The remaining board or committee members shall decide if a conflict of interest exists.
\item \emph{Procedures for Addressing the Conflict of Interest}
\begin{enumerate}
\item An interested person may make a presentation at the governing board or committee meeting, but after the presentation, they shall leave the meeting during the discussion of, and the vote on, the transaction or arrangement involving the possible conflict of interest.
\item The chair of the governing board or committee shall, if appropriate, appoint a disinterested person or committee to investigate alternatives to the proposed transaction or arrangement.
\item After exercising due diligence, the governing board or committee shall determine whether the Organization can obtain with reasonable efforts a more advantageous transaction or arrangement from a person or entity that would not give rise to a conflict of interest.
\item If a more advantageous transaction or arrangement is not reasonably possible under circumstances not producing a conflict of interest, the governing board or committee shall determine by a majority vote of the disinterested directors whether the transaction or arrangement is in the Organization's best interest, for its own benefit, and whether it is fair and reasonable. In conformity with the above determination it shall make its decision as to whether to enter into the transaction or arrangement.
\end{enumerate}
\item \emph{Violations of the Conflicts of Interest Policy}
\begin{enumerate}
\item If the governing board or committee has reasonable cause to believe a member has failed to disclose actual or possible conflicts of interest, it shall inform the member of the basis for such belief and afford the member an opportunity to explain the alleged failure to disclose.
\item If, after hearing the member's response and after making further investigation as warranted by the circumstances, the governing board or committee determines the member has failed to disclose an actual or possible conflict of interest, it shall take appropriate disciplinary and corrective action.
\end{enumerate}
\end{enumerate}


\subsection*{Section 4. Records of Proceedings}

The minutes of the governing board and all committees with board delegated powers shall contain:

\begin{enumerate}
\item The names of the persons who disclosed or otherwise were found to have a financial interest in connection with an actual or possible conflict of interest, the nature of the financial interest, any action taken to determine whether a conflict of interest was present, and the governing board's or committee's decision as to whether a conflict of interest in fact existed.
\item The names of the persons who were present for discussions and votes relating to the transaction or arrangement, the content of the discussion, including any alternatives to the proposed transaction or arrangement, and a record of any votes taken in connection with the proceedings.
\end{enumerate}

\subsection*{Section 5. Compensation}

\begin{enumerate}
\item A voting member of the governing board who receives compensation, directly or indirectly, from the Organization for services is precluded from voting on matters pertaining to that member's compensation.
\item A voting member of any committee whose jurisdiction includes compensation matters and who receives compensation, directly or indirectly, from the Organization for services is precluded from voting on matters pertaining to that member's compensation.
\item No voting member of the governing board or any committee whose jurisdiction includes compensation matters and who receives compensation, directly or indirectly, from the Organization, either individually or collectively, is prohibited from providing information to any committee regarding compensation.
\end{enumerate}

\subsection*{Section 6. Annual Statements}

Each director, principal officer and member of a committee with governing board delegated powers shall annually sign a statement which affirms such person:

\begin{enumerate}
\item Has received a copy of the conflicts of interest policy,
\item Has read and understands the policy,
\item Has agreed to comply with the policy, and
\item Understands the Organization is charitable and in order to maintain its federal tax exemption it must engage primarily in activities which accomplish one or more of its tax-exempt purposes.
\end{enumerate}

\subsection*{Section 7. Periodic Reviews}

To ensure the Organization operates in a manner consistent with charitable purposes and does not engage in activities that could jeopardize its tax-exempt status, periodic reviews shall be conducted. The periodic reviews shall, at a minimum, include the following subjects:

\begin{enumerate}
\item Whether compensation arrangements and benefits are reasonable, based on competent survey information, and the result of arm's length bargaining.
\item Whether partnerships, joint ventures, and arrangements with management organizations conform to the Organization's written policies, are properly recorded, reflect reasonable investment or payments for goods and services, further charitable purposes and do not result in inurement, impermissible private benefit or in an excess benefit transaction.
\end{enumerate}

\subsection*{Section 8. Use of Outside Experts}

When conducting the periodic reviews as provided for in Article VII, the Organization may, but need not, use outside advisors. If outside experts are used, their use shall not relieve the governing board of its responsibility for ensuring periodic reviews are conducted.

\section*{ARTICLE X. INDEMNIFICATION}


\subsection*{Section 1. General}

To the full extent authorized under the laws of Florida, the corporation shall indemnify any director, officer, employee, or agent, or former member, director, officer, employee, or agent of the corporation, or any person who may have served at the corporation's request as a director or officer of another corporation (each of the foregoing members, directors, officers, employees, agents, and persons is referred to in this Article individually as an “indemnitee”), against expenses actually and necessarily incurred by such indemnitee in connection with the defense of any action, suit, or proceeding in which that indemnitee is made a party by reason of being or having been such member, director, officer, employee, or agent, except in relation to matters as to which that indemnitee shall have been adjudged in such action, suit, or proceeding to be liable for negligence or misconduct in the performance of a duty. The foregoing indemnification shall not be deemed exclusive of any other rights to which an indemnitee may be entitled under any bylaw, agreement, resolution of the Board of Directors, or otherwise.

\subsection*{Section 2. Expenses}

Expenses (including reasonable attorneys' fees) incurred in defending a civil or criminal action, suit, or proceeding may be paid by the corporation in advance of the final disposition of such action, suit, or proceeding, if authorized by the Board of Directors, upon receipt of an undertaking by or on behalf of the indemnitee to repay such amount if it shall ultimately be determined that such indemnitee is not entitled to be indemnified hereunder.

\subsection*{Section 3. Insurance}

The corporation may purchase and maintain insurance on behalf of any person who is or was a member, director, officer, employee, or agent against any liability asserted against such person and incurred by such person in any such capacity or arising out of such person's status as such, whether or not the corporation would have the power or obligation to indemnify such person against such liability under this Article.

\section*{ARTICLE XI. BOOKS AND RECORDS}

The corporation shall keep complete books and records of account and minutes of the proceedings of the Board of Directors.

\section*{ARTICLE XII. AMENDMENTS}


\subsection*{Section 1. Articles of Incorporation}

The Articles may be amended in any manner at any regular or special meeting of the Board of Directors, provided that specific written notice of the proposed amendment of the Articles setting forth the proposed amendment or a summary of the changes to be effected thereby shall be given to each director at least three days in advance of such a meeting if delivered personally, by facsimile, or by e-mail or at least five days if delivered by mail. As required by the Articles, any amendment to Article III or Article VI of the Articles shall require the affirmative vote of all directors then in office. All other amendments of the Articles shall require the affirmative vote of an absolute majority of directors then in office.

\subsection*{Section 2. Bylaws}

The Board of Directors may amend these Bylaws by majority vote at any regular or special meeting. Written notice setting forth the proposed amendment or summary of the changes to be effected thereby shall be given to each director within the time and the manner provided for the giving of notice of meetings of directors.
ADOPTION OF BYLAWS

ADOPTED AND APPROVED by the Board of Directors on this 24th Day of April 2024.

\end{document}
